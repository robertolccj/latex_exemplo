\documentclass[12pt]{article}

\usepackage[T1]{fontenc} % Pacote para codificação da fonte;
\usepackage[utf8]{inputenc} % Pacote para codificação do "texto puro";
\usepackage[brazil]{babel} % Pacote para reconhecer caracteres do idioma brasileiro;
\usepackage{csquotes} % Pacote requerido pelo pacote 'babel';
\usepackage{microtype}  % Pacote para melhorar a 'justificação' do texto;
\usepackage{lmodern}  % Pacote para usar a fonte 'Latin Modern'.
\usepackage{graphicx} % Pacote para uso de figuras (jpg, png, pdf, etc)
\usepackage{float} % Para usar figura com argumento "H" (here);
\usepackage{enumitem} % Para melhorar itemização;
\usepackage[hidelinks]{hyperref} % Links "clicáveis" (citações, bibliografia);
\usepackage{booktabs} % Top and bottom rules for table;
\usepackage{authblk}  % Autor e Afiliação (instituição) do artigo;
%\usepackage[sorting=none, style = abnt]{biblatex} % Pacote p/ geração de bibliografia;
%\addbibresource{bibliografia.bib} % Arquivo de bibliografia;



\usepackage{draftwatermark}
\SetWatermarkText{RASCUNHO}
\SetWatermarkScale{0.8}










%===============================================================================
% FORMATAÇÃO (principais características brasileiras)
%===============================================================================
\usepackage[a4paper,includeheadfoot]{geometry}
\geometry{
    %showframe,
    tmargin=1cm, % + head
    bmargin=0.6cm, % +footskip
    lmargin=2cm,
    rmargin=2cm,
    headsep=0.1cm,
    footskip=0.6cm
} 

\usepackage{helvet} %Fonte compatível com a Arial;
\renewcommand{\familydefault}{\sfdefault} % Fonte sem serifa

\usepackage[parfill]{parskip} % Sem identar linhas.

\linespread{1.25} % Espaçamento entre linhas {1.25}*1.2 = 1.5, como no word;

%===============================================================================













% =============================================
% Dados para: título, autores, instituições; 
% =============================================
\title{Título do Projeto 2}

\author{Nome do Autor}
\affil{Nome do Curso}


%\author{Participante \thanks{email}}
%\affil{Curso, IFCE, Campus XXXXXX}


%\date{Novembro de 2022}
\date{\today}
% =============================================















%================================================
% A parte textual do documento inicia aqui;
%================================================
\begin{document}
	
\maketitle
	
%\begin{abstract}
%\noindent Resumo do artigo.
%\end{abstract}













\section{Apresentação}

Quando todo o projeto for descrito, fazer aqui uma espécia de resumo.





\begin{description}[leftmargin=0.0ex]

\item[Comunidade externa beneficiada]
Professores e alunos da rede municipal de ensino (pública e privada).

\item[Comunidade interna beneficiada]
Estudantes e professores dos cursos de licenciatura em física.

\end{description}











\section{Justificativa}


\begin{itemize}[leftmargin=0.0ex,itemsep=0pt,parsep=0pt,partopsep=0pt] 
\item Problemática 1;
\item Problemática 2;
\item Problemática 3;
\end{itemize}












\section{Objetivos}

\subsection{Objetivo geral}
Este aqui é o objetivo geral.


\subsection{Objetivo específico}
Aqui insere-se os objetivos específicos.


















\section{Metodologia}

% Qual a base do projeto?
Qual a base do projeto?

% Avaliação
Como será feita a avaliação?



% Como a comunidade irá participar?
Como a comunidade irá participar?













% -- FIM --

\end{document}
